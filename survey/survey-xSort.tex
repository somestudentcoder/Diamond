\chapter{xSort}

\label{chap:tool}
xSort is an offline card sorting tool which only provides a local version for macOS. In contrast to 
other modern tools it is not up to modern standards. Both the design and the handling are no
longer up to date. Because of being free of charge it is still an alternative for local card sorting on 
macOS. Furthermore also the few existing analysis tools can be used completely free of charge~\parencite{xSort}.

\begin{figure}[h] 
\centering
\includegraphics[keepaspectratio,width=\linewidth,height=\halfh]{images/xsort-sorting.png}
\caption[xSort Application] { This screenshot shows the sorting in the table view of xSort.
\imgcredit{Screenshot was captured by Markus Stradner using
\textcite{xSort} on macOS Catalina 10.15.7, 28.11.2020.} }
\label{fig:xSort-sorting}
\end{figure}


\section{Business Model}
The tool is fully free of charge, so no features are hidden behind a paywall. It was created by the 
Enough Pepper company and is currently not maintained. As it is a local program you don't need 
to create an account.

\section{Card Sorting}
The setup process of a card sorting is quite simple and works well. All the settings can be made 
on one site. You can also type in welcome and thank you messages. The cards and categories 
(for closed card sorting) can be imported via .csv file. Also names for different user profiles can be 
set.

\begin{figure}[h] 
\centering
\includegraphics[keepaspectratio,width=460px]{images/xsort-setup.png}
\caption[xSort Setup] { This screenshot shows the setup view of xSort.
\imgcredit{Screenshot was captured by Markus Stradner using
\textcite{xSort} on macOS Catalina 10.15.7, 28.11.2020.} }
\label{fig:xSort-setup}
\end{figure}

In the exercise mode which is the mode to run through the study you have to select your profile, 
your gender and have to type in your age. After that you can change between 
the table view and an outline view during the whole sorting. In the table view you have to pick up 
the cards in the bottom right corner and assign it to the desired group. If you would like to use the 
outline view you have to drag the card items on the left side to the desired group item on the right side. 

At an open card sorting you have to create the categories yourself, which is also not so 
convenient. Because there is no auto-alignment of the groups and the cards, the clean up 
tool exists to align all the elements on a grid via the gearwheel button where you can also restart 
the card sorting. 

xSort also offers a zooming tool to zoom in and out
of the grouping area. But this is not very practical, because you have troubles with scrolling and 
seeing all the elements if you zoom in. 

One feature no other tool has is that it is possible to group in sub-categories. This is allowed in 
both the table view and the outline view. Sometimes errors occur during sorting. Then no more 
cards can be moved. This occurs especially in the outline view.

\begin{figure}[h] 
\centering
\includegraphics[keepaspectratio,width=460px]{images/xsort-outlineview.png}
\caption[xSort Outline View] { This screenshot shows the outline view of xSort.
\imgcredit{Screenshot was captured by Markus Stradner using
\textcite{xSort} on macOS Catalina 10.15.7, 28.11.2020.} }
\label{fig:xSort-outlineview}
\end{figure}

\section{Analytics}
The tool also includes a few analysis tools. But the extent of these is not comparable to that of 
OptimalSort for example. It just offers a cluster tree, general statistics like unclassified cards, 
number and average duration of the participations and a distance table.  

With xSort, however, the age, gender and profile can also be taken into account in the evaluation. 
You can filter the results by those user attributes and create a report accordingly, which you can 
then save as a PDF or print directly. But also in that area the design is very bad and except the 
cluster tree no graphics are displayed.

For generation of such a report you have to choose the desired components. You can select the f
ollowing ones: Problem Information, cards, groups, profiles, sessions, unclassified cards, cluster 
tree, marks. If you would like to generate a cluster tree you can select a single linkage, average 
linkage or complete linkage and you can also differ to flatten groups or use sub groups.

\begin{figure}[h] 
\centering
\includegraphics[keepaspectratio,width=460px]{images/xsort-analysis.png}
\caption[xSort Analysis] { This screenshot shows a cluster tree in the results section of xSort.
\imgcredit{Screenshot was captured by Markus Stradner using
\textcite{xSort} on macOS Catalina 10.15.7, 28.11.2020.} }
\label{fig:xSort-setup}
\end{figure}

\section{Summary \& Ratings}
In comparison with more modern card sorting tools like OptimalSort, in xSort there is a password 
security possible, it has a zooming tool, a clean up tool and the biggest advantage is, that it’s free 
of charge.

The main disadvantages are that it can’t be used for online studies, the design is very bad and not 
up to date, there are also small bugs in it. The documentation stuff can only be found online and  
while assigning the cards, the groups don’t align with the grid. So there is a mess after assigning 
more cards and you have to use the clean up tool very often.

\newpage

\begin{table}[t]
\centering
\begin{tabularx}
{\linewidth}{|l|X|}
\hline \textbf{Feature/Characteristic} & \textbf{Availability in xSort} \\ 
\hline Card Sorting & Open and closed. \\ 
\hline Card Limit & No. \\
\hline Participant Limit & No. \\
\hline Analytics & Cluster tree, general statistics and distance table. \\ 
\hline Documentation & Instructions can be found online. \\
\hline Business Model & Fully free of charge. \\
\hline Import formats & .xml, .csv \\ 
\hline Export formats & .xml, .csv, .html, .pdf \\ 
\hline Sub-Categories & Yes. \\ 
\hline Playback of user-sessions & No. \\ 
\hline Data preparation & Only by hand through export reports. \\ 
\hline
\end{tabularx} 
\caption[Feature summary of xSort] 
{ 
This table summarizes all the features and characteristics of xSort
to provide an easy to read overview.
}
\label{tab:features-xSort}
\end{table}

For a quick overview we agreed on four ratings for the tool. The ratings can be
found in Table~\ref{tab:rating-xSort} and range from 0-5.

\begin{table}[h] 
\centering 
\begin{tabularx}{\linewidth}{|X|X|X|X|X|}
\hline
Simplicity & Documentation & Features & Business Model & Average \\ 
\hline 
2 & 2 & 3 & 4 & 2.75 \\ 
\hline 
\end{tabularx} 
\caption[Ratings for xSort] {
Ratings for xSort including the average rating.
} 
\label{tab:rating-xSort}
\end{table}