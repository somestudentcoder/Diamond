\chapter{Introduction}

\label{chap:Intro}

Before comparing the different tools we want to define the differences between 
different card sortin types and their different challenges. Furthermore, this 
chapter will also include an overview of all the reviewed tools as well as a 
detailed description of our review process and which features we focused on.


\section{Card Sorting}

Card sorting is a type of usability test in which participants are
asked to sort cards into different categories. Usually a distinction
is made between open, closed, and hybrid card sorting. Open card
sorting does not provide any predefined categories, and instead users
create their own categories. Closed card sorting gives the
participants a predefined list of categories into which the cards have
to be sorted. Hybrid card sorting is a mix of the two previous models,
where some categories are predefined, but users can create more. This
survey focuses on the capabilities of tools to work with open card
sorting studies.

Open card sorting comes with some additional challenges, since the
number of categories created by participants is not limited, and in
some instances sub-categories may be permitted. Furthermore, many of
the categories defined by users will be extremely similar or the same,
but may use different words to describe the same thing. These
categories then have to be merged in an additional step before drawing
any meaningful conclusions.


\section{Tools}

This chapter lists the tools in the card sorting field that were
reviewed in the process of this survey. These tools can be further
grouped by their functionalities. One tool, kardSort, is only intended
for the card sorting and offers no analytics at all. Some other tools
are only able to analyze previous card sort experiments, but offer no
sorting capabilities. Finally the remaining tools offer both sorting
and analytic capabilities. An overview can be seen in
Table~\ref{tab:reviewed-tools}.

\begin{table}[tp]
\centering
\begin{tabularx}
{\linewidth}{|X|X|X|}
\hline \textbf{Card Sorting} & \textbf{Analytics} & \textbf{Both}\\ 
\hline kardSort & SynCaps v3 & UXtweak \\ 
\hline & Casolysis & ProvenByUsers \\
\hline & CSA & xSort \\
\hline & & OptimalSort \\
\hline
\end{tabularx} 
\caption[Reviewed Tools] 
{ 
This table summarizes all the tools that were reviewed and categorizes
them by functionality.
}
\label{tab:reviewed-tools}
\end{table}


More tools were previewed before coming to a decision as to which
tools will be in the final survey. There were tools we were not able
to review as they are no longer available or have been acquired by
competitors. One tool, ~\textcite{UserZoom}, offers no free version
and did not answer to our inquiry for a review version. We listed all
tools that we looked at but did not review below:

\begin{itemize}
    \item ~\parencite{SortIt} (not buildable)
    \item ~\parencite{UserZoom} (did not respond to inquiry)
    \item UsabilityTools UXSuite (acquired by competitor)
    \item ~\parencite{SimpleCardSort} (discontinued)
    \item ~\parencite{usabiliTest} (acquired by competitor)
    \item ConceptCodify (acquired by competitor)
\end{itemize}


\section{Review Process} The reviewing of tools was done in parallel
by three different  reviewers. In order to ensure streamlined reviews
and produce comparable results a review process was defined. Each
reviewer had to perform the same tasks and check for certain
pre-defined features. These features are listed below:

\begin{itemize}
    \item Participant limit
    \item Card limit
    \item Business model
    \item Availability of documentation
    \item Support for sub-categories
    \item Possibility for questionnaires
    \item Import data capabilities
    \item Export data capabilities
    \item User session playbacks
    \item Data preparation capabilities
    \item Types of analytical visualizations
\end{itemize}

Once a review had taken place. It was discussed
in a meeting with the other reviewers and ratings were assigned in
accordance with all members.

Each tool was tested with 2 datasets made up of two sets of cards. One
dataset is made up of 10 animals and the second is a list of 55 car
manufacturers. The cards were then used in open card sorting
experiments by the reviewers. The datasets can be viewed in Table
~\ref{tab:dataset-animal} and ~\ref{tab:dataset-car}.

\begin{table}[tp]
\centering
\begin{tabularx}
{\linewidth}{|X|X|X|X|X|}
\hline
Dog& Elephant  & Whale     & Lion& Mouse         \\
\hline
Cow       & Bear& Butterfly & Snake  & Badger    \\
\hline
\end{tabularx} 
\caption[Animal Dataset] 
{ 
This table lists all the cards in the animal dataset.
}
\label{tab:dataset-animal}
\end{table}

\begin{table}[bp]
\centering
\begin{tabularx}
{\linewidth}{|X|X|X|X|X|}
\hline
Volkswagen    & BYD      & Ford         & Infiniti    & Rimac   \\
\hline
Kia           & Skoda    & Cadillac     & Mitsubishi  & Jeep    \\
\hline
Honda         & Bugatti  & Jaguar       & Dodge       & Lincoln \\
\hline
Toyota        & Citroen  & Aston Martin & GMC         & Smart   \\
\hline
Porsche       & Hyundai  & Chrystler    & Lexus       & Tesla   \\
\hline
Seat          & Nissan   & Subaru       & Range Rover & Lancia  \\
\hline
Mazda         & Saab     & Lamborghini  & Rolls-Royce & Dacia   \\
\hline
BMW           & Renault  & Koenigsegg   & Volvo       & Lotus   \\
\hline
Audi          & Peugeot  & Alfa Romeo   & Suzuki      & Mini    \\
\hline
KTM           & Ferrari  & Fiat         & Bentley     & Lada    \\
\hline
Mercedes-Benz & Chevrolet& Maserati     & McLaren     &        \\
\hline
\end{tabularx} 
\caption[Car Dataset] 
{ 
This table lists all the cards in the car manufacturer dataset.
}
\label{tab:dataset-car}
\end{table}

