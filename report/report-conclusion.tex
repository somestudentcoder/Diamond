\chapter{Conclusion}

\label{chap:Concl}

Now that all the details of the implementation of Diamond have been covered, this chapter conlcludes the report and ends with some possible improvement for the future and some final remarks.

\section{Possible Improvements}

Due to limited time, some features, which are present in many other card sorting applications, were not implemented. Here is a short list of features we thought of:

\begin{itemize}
  \item Currently, the only information on the participants that is stored is their name. It could be interesting to give the creator of a study the possibility to create custom questionaires to extract more information about the participants
  \item It would be nice to allow users to decide whether they want to create an open, close or hybrid card sorting study. This could be done with only minor additional effort. 
  \item While an overview of the results is provided within the app, for most use cases it is necessary to export the results and then import them into some third application for further evaluation. The tracking and display of statistics such as average sorting duration or how often users changed their minds when sorting cards would be quite helpful and interesting.
  \item It would be possible create sub-groups when sorting the cards. This would require rescaling the size of the groups during the sorting process and abstracting the cards into another component.
\end{itemize}



\section{Final Remarks}

As of the time of writing this report, Diamond is hosted publicly on Heroku under \url{https://iaweb-diamond.herokuapp.com/#/tests}.

The git repository is \url{https://github.com/somestudentcoder/Diamond}.


