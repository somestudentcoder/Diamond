\chapter{Technologies}
\label{chap:Tech}
\TODO{WRITE Intro}

\section{Angular}

\TODO{WRITE}

\section{Node.js}

\TODO{WRITE}

\section{MongoDB}

\TODO{WRITE}

\section{Heroku}

Heroku was used for online hosting. In general it is a container-based 
cloud Platform as a Service (PaaS). This means that everything about 
setting up a server, online deployment and hosting is provided by it. It 
is very intuitive to use and free of charge, but an account is needed. 

Heroku offers to connect a GitHub repository to that account. This is 
quite convenient, because you can choose a branch to enable 
auto-deployment after every single push on this branch. For this reason 
you can keep your web application up-to-date very easily. 

It is recommended to created an Heroku branch and to push on that 
branch, when there is a larger number of changes to deploy online. 
And the main branch should just be configured with the localhost 
settings for testing locally.

